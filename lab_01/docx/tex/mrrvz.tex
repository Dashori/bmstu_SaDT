% !TeX document-id = {c68f4be8-c497-43e0-82df-e9ebfbea9577}
% !TeX TXS-program:pdflatex = pdflatex -synctex=1 -interaction=nonstopmode --shell-escape %.tex
% новая команда \RNumb для вывода римских цифр
\documentclass[a4paper,12pt]{article}
\usepackage{amssymb}
\usepackage{amsmath}
\usepackage{amsthm} 
\usepackage{caption}
\usepackage{misccorr}
\usepackage[noadjust]{cite}
\usepackage{cmap} 
\usepackage[utf8]{inputenc}
\usepackage[T2A]{fontenc}
\usepackage[english, russian]{babel}
\usepackage{graphics}
\usepackage{graphicx}
\usepackage{textcomp}
\usepackage{verbatim}
\usepackage{makeidx}
\usepackage{geometry}
\usepackage{float}
\usepackage{bm}
\usepackage{mathtools}
\usepackage{graphicx}
\usepackage{listings}
\usepackage{courier}
\usepackage{graphicx}
\usepackage[table]{xcolor}
\usepackage{color}
\lstset{basicstyle=\fontsize{10}{10}\selectfont,breaklines=true}

\newcommand{\specchapter}[1]{\chapter*{#1}\addcontentsline{toc}{chapter}{#1}}
\newcommand{\specsection}[1]{\section*{#1}\addcontentsline{toc}{section}{#1}}
\newcommand{\specsubsection}[1]{\subsection*{#1}\addcontentsline{toc}{subsection}{#1}}
\newcommand{\RNumb}[1]{\uppercase\expandafter{\romannumeral #1\relax}}
\newcommand{\jj}{\righthyphenmin=20 \justifying}


% геометрия
\geometry{pdftex, left = 2cm, right = 2cm, top = 2.5cm, bottom = 2.5cm}

\setcounter{tocdepth}{4} % фикс переноса 
\righthyphenmin = 2
\tolerance = 2048

\begin{document}
	\thispagestyle{empty}
	
	\noindent \begin{minipage}{0.15\textwidth}
		\includegraphics[width=\linewidth]{b_logo}
	\end{minipage}
	\noindent\begin{minipage}{0.9\textwidth}\centering
		\textbf{Министерство науки и высшего образования Российской Федерации}\\
		\textbf{Федеральное государственное бюджетное образовательное учреждение высшего образования}\\
		\textbf{«Московский государственный технический университет имени Н.Э.~Баумана}\\
		\textbf{(национальный исследовательский университет)»}\\
		\textbf{(МГТУ им. Н.Э.~Баумана)}
	\end{minipage}
	
	\noindent\rule{18cm}{3pt}
	\newline\newline
	\noindent ФАКУЛЬТЕТ \text{«Информатика и системы управления»} \newline\newline
	\noindent КАФЕДРА \text{«Программное обеспечение ЭВМ и информационные технологии»}\newline\newline
	\newline
	
	
	\begin{center}
		\noindent\begin{minipage}{1.0\textwidth}\centering
			\large\textbf{ ОТЧЕТ ПО ЛАБОРАТОРНОЙ РАБОТЕ №1}\\
			\large\textbf{ «ДЛИННАЯ АРИФМЕТИКА»}\\
			\large\textbf{  по курсу «Типы и структуры данных»}\\
		\end{minipage}
	\end{center}
	""\newline\newline\newline\newline

	\noindent Студент\text{~~~~~~~~~~~~~~~Чепиго Дарья Станиславовна~~~~~~~~~~~~~~~~~~~~~~~~}\newline
	Группа{~~~~~~~~~~~~~~~~~ИУ7-44Б~~~~~~~~~~}\newline\newline\newline\newline\newline\newline\newline
	
	
	\noindent\begin{tabular}{lcc}
	Студент: ~~~~~~~~~~~~~~~~~~~~~~~~~~~~~~~~~~~~~~~~~~~~~~~~~~~~~~~~~& $\underline{\text{~~~~~~~~~~~~~~~~}}$ & $\underline{\text{~~Чепиго Д.С..~~}}$ \\
	& \footnotesize подпись, дата  & \footnotesize Фамилия, И.О. \\
	& &  \\
	Преподаватель: & $\underline{\text{~~~~~~~~~~~~~~~~~}}$ & $\underline{\text{~~~~ Барышникова М.Ю.~~~}}$ \\
		& \footnotesize подпись, дата & \footnotesize Фамилия, И. О. \\
	\end{tabular}
	
	""\newline\newline\newline\newline\newline
	
	\noindent ~Оценка $\underline{\text{~~~~~~~~~~~~~~~~~~~~~~~~~~~~~~~~~~~~}}$
	""\newline
	
	\begin{center}
		\vfill
		Москва -- 2021
		~г.
	\end{center}
	\clearpage
	
\end{document}
